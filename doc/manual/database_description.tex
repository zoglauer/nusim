%\documentclass[12pt]{article}

%\usepackage[pdftex]{graphicx}

%\begin{document}



\chapter{NuSIM Observatory Alignement Database}
%\author{Isaac Sheff}
%\maketitle
%\newpage
%\tableofcontents
%\newpage
\section{Purpose}
The various components of NuSIM each operate in their own coordinate systems, and so in order to simulate the entirety of the spacecraft, it is necessary to keep track of the transformations between these systems. Furthermore, for many tests it is necessary to simulate NuSIM under changing conditions, and so for each timestep a set of coordinate system transformations must be provided. 

\subsection{List of Coordinate System Transformations}


\begin{tabular}{| r | l |}
\hline
{\bf Name}	&	{\bf Transformation}						\\
\hline
Inertial-SC	&	Inertial Frame to Spacecraft				\\
SC-FB		&	Spacecraft to Focal Bench				\\
FB-FP0		&	Focal Bench to First Focal Plane Module		\\
FB-FP1		&	Focal Bench to Second Focal Plane Module	\\
FB-MD0		&	Focal Bench to First Metrology Detector		\\
FB-MD1		&	Focal Bench to Second Metrology Detector	\\
FB-AS0		&	Focal Bench to First Aperture Stop			\\
FB-AS1		&	Focal Bench to Second Aperture Stop		\\
FB-OB		&	Focal Bench to Optics Bench (the mast)		\\
OB-OM0		&	Optics Bench to First Optics Module			\\
OB-OM1		&	Optics Bench to Second Optics Module		\\
OB-ML0		&	Optics Bench to First Metrology Laser		\\
OB-ML1		&	Optics Bench to Second Metrology Laser		\\
OB-ST		&	Optics Bench to Star Tracker				\\
\hline
\end{tabular}


\section{Format}
In order to provide NuSIM with a set of coordinate system transformations for each time step, a comma separated values (.csv) spreadsheet is used.
\subsection{Layout}
Each time-step is a pair of sequential rows: one for the translation between the coordinate systems, presented in (x, y, z), and the other for the rotation, presented as a quaternion, with the real component last: (q1, q2, q3, q4). Thus each coordinate system transformation is a set of four adjacent columns, and on the first row of each time-step the fourth column of each coordinate system transformation is empty. 

There are several rows before the time-steps begin, representing header data at the top of the database file. These will include a row in which the name of each coordinate system transformation is in the first of the four columns its values occupy. 

There are no rows between the time-steps, and no rows of interest after the time-steps. 

\subsection{Template}
With each database format revision, a template file is produced, often in microsoft excel format. This file is simply a database with a single time-step: all the components in their ideal configuration. Historically, these file have been named along the lines of ``NuSIM\_OrientationsIDEAL\_005.xls." Note that while Microsoft Excel's exported CSV files use carriage return (``\textbackslash r") line-breaks, NuSIM requires its input databases to use newline (``\textbackslash n") line-breaks. 

\subsection{Diagrams}
\subsubsection{File Layout}
\begin{tabular}{| c | c | c | c | c |}
\hline
\multicolumn{5}{| c |}{ }\\
\multicolumn{5}{| c |}{\Large Header}\\
\multicolumn{5}{| c |}{ }\\
\hline
 & & & & \\
{extra header info} & Inertial-SC & SC-FB & {\Huge \ldots} & OB-ST\\
 & & & & \\
{\Huge $\Downarrow$} & {\Huge $\Downarrow$} & {\Huge $\Downarrow$} & {\Huge $\Downarrow$} & {\Huge $\Downarrow$}\\
  & & & & \\
\hline
\end{tabular}

\subsubsection{column layout}
\begin{tabular}{| c || l | l | l | l |}
\hline
{\bf Section} & \multicolumn{4}{ c |}{\bf Content}\\
\hline
 & \multicolumn{4}{ c |}{ }\\
 & \multicolumn{4}{ c |}{ header info}\\
 & \multicolumn{4}{ c |}{ }\\
\cline{2-5}
Header & Name & & & \\
\cline{2-5} 
 & \multicolumn{4}{ c |}{ }\\
 & \multicolumn{4}{ c |}{ more header info}\\
 & \multicolumn{4}{ c |}{ }\\
\hline
Time- & $x$  translation & $y$ translation & $z$ translation & \\
\cline{2-5}
Step 0 & $q_1$ ($i$ coefficient) & $q_2$ ($j$ coefficient) & $q_3$ ($k$ coefficient) & $q_4$ (real)\\
\hline
Time- & $x$  translation & $y$ translation & $z$ translation & \\
\cline{2-5}
Step 1 & $q_1$ ($i$ coefficient) & $q_2$ ($j$ coefficient) & $q_3$ ($k$ coefficient) & $q_4$ (real)\\
\hline
Time- & $x$  translation & $y$ translation & $z$ translation & \\
\cline{2-5}
Step 2 & $q_1$ ($i$ coefficient) & $q_2$ ($j$ coefficient) & $q_3$ ($k$ coefficient) & $q_4$ (real)\\
\hline
 & & & & \\
 \multicolumn{1}{| c ||}{\Huge $\Downarrow$} & \multicolumn{1}{ c |}{\Huge $\Downarrow$} &  \multicolumn{1}{ c |}{\Huge $\Downarrow$} &  \multicolumn{1}{ c |}{\Huge $\Downarrow$} &  \multicolumn{1}{ c |}{\Huge $\Downarrow$}\\
  & & & & \\
  \hline
\end{tabular}

\section{Thermal Mast Bending}

The document ``Thermal\_Distortion\_2\_for\_JPL.xlsx" details the effects on the mast under the thermal effects of sunlight. In particular, it provides $x$ and $y$ offsets of the focal plane intersections with the beams from the optics, as well as $rotZ$, the twisting of the mast itself. These effects are most pronounced for a 170 degree angle of the mast toward the sun. To more fully model the effects of thermal mast bending on NuSTAR, a coordinate system transformation database had to be constructed for the mast in the configurations predicted for a full orbit. 


\subsection{Thermal mast bending database}
The thermal mast bending databases is of the same format as the main ideal alignment database. It is produced by an IDL program which reads the simulated mast pointings and derives a transformation. Details of the mast databases and their accuracy see Appendix \ref{mast}.

