\chapter{The various input file formats}

\section{For the source engine}

\subsection{fits-files}

NuSIM is able to read standard (Chandra) fits file as input.
You can test if NuSIM is able to read the file with the program ViewFits, which is located in resource/examples/AddOn. It is not compiled by default, thus you have to type
\begin{verbatim}
make PRG=ViewFits only
\end{verbatim}
To run it type:
\begin{verbatim}
ViewFits -f <you fits file here>
\end{verbatim}
If a canvas with your image shows up, you are good to go.

\subsection{The spe-format for input spectra \label{label:spectrumfile}}

This format represents an input spectrum.
It looks like this:
\begin{verbatim}
IP LIN

DP 3 1.775
DP 3.04907 1.73616
DP 3.09894 1.69814
DP 3.14962 1.66093
DP 3.20114 1.62449
DP 3.25349 1.58882
DP 3.30671 1.55391
DP 3.36079 1.51972
DP 3.41576 1.48626
DP 3.47163 1.4535
DP 3.52841 1.42144

EN
\end{verbatim}

The IP line give the interpolation type. Possible are LINLIN (equals LIN), LINLOG, LOGLIN, and LOGLOG (equals LOG). Choose the interpolation type which is most reasonable for you data (i.e in which your looks like a straight line).
If your spectrum contains line features it is best to always use LIN.

Data point are represented by the "DP" keyword.
The first number represents the energy and the second the flux in something like ph/cm$^2$/s/keV.
The normalization does not need to be correct because it is given with the "Flux" keyword in the source input.

Make sure the last line in your file is "EN" for "The End".

Lines starting with "\#" are interpreted as comments

\subsection{The lgt-format for input light curves \label{label:lightcurvefile}}

This format is basically identical to the one with a spectrum, only the second column is now a time in seconds
It looks like this:
\begin{verbatim}
IP LINLIN

DP 0.0 0.1
DP 0.1 0.1
DP 0.2 0.2
DP 0.3 0.4
DP 0.4 0.8
DP 0.5 0.4
DP 0.6 0.2
DP 0.7 0.1
DP 0.8 0.1
DP 0.9 0.1
DP 1.0 0.1
EN
\end{verbatim}

The IP line give the interpolation type. Possible are LINLIN (equals LIN), LINLOG, LOGLIN, and LOGLOG (equals LOG). Choose the interpolation type which is most reasonable for you data (i.e in which your looks like a straight line).
If your spectrum contains line features it is best to always use LIN.

Data point are represented by the "DP" keyword.
The first number represents the time and the second the light curve value in arbitrary units.
The normalization does not need to be correct because it is given with the "Flux" keyword in the source input.

Make sure the last line in your file is "EN" for "The End".

Lines starting with "\#" are interpreted as comments

\subsection{The 3Ddat format \label{input:3Ddat}}

The 3Ddat format represents a 3D data space spanned by RA, DEC, and energy. It's content is flux at the axis position in ph/cm$^2$/s/keV/sr.

The file look like this:

\begin{verbatim}
IP LIN

# RA axis in deg:
PA 35.1 35.2 35.3 35.4 35.5
# Dec axis in deg:
TA 5.05 5.10 5.15 5.20 5.25
# Energy axis in keV:
EA 10 15 20 25 30 35 40

AP 0 0 0  0.50
AP 0 0 1  0.25
AP 0 0 2  0.12
AP 0 0 3  0.07

# Skip the rest

EN
\end{verbatim}

The IP line give the interpolation type. Currently only LIN, linear interpolation, is supported.

The next three lines represent the data points on the axes at which the flux is given. PA (phi-axis) represents the right ascension in degree, TA (theta-axis) represents the declination in degree, EA (energy-axis) represents the energy in keV.

The following section gives the value at the axis points for the given ID (number starting with zero!) of the data point on the three axis.
For example "AP 3 1 5 1.6" represents a flux of 1.6 ph/cm$^2$/s/keV/sr for the 4th axis point in the RA-axis (35.4 deg), the 2nd axis point in the DEC-axis (5.10 deg), and the 6th axis point of the energy axis (35 keV).

Make sure the last line in your file is "EN" for "The End".

Lines starting with "\#" are interpreted as comments


\subsection{The src file to import sources}

Finally it is possible to import whole source lists into the source engine.
This is done via the src-file. the following is an example:
{\footnotesize
\begin{verbatim}
Mono          1  266.20  -29.00   1  40.0                     0.00001            1
Linear        1  266.25  -29.00   2  30.0  35.0               0.00002            1
Plaw          1  266.30  -29.00   3  3.0  80.0   0.3          0.00003            1
BrokenPlaw    1  266.40  -29.00   4  5.0  80.0  20.0 0.6 2.4  0.00004            1
FileDiffFlux  1  266.35  -29.00   5  $(NUSIM)/resource/configs/SourceGen.dat 0.4 1       
BlackBody     1  266.35  -29.05   6  5.0  20.0 80.0           0.00004            1 
Absorbed      1  266.30  -29.05   7  3 82 7e-6*pow(x/10,-1.1)*exp(-sqrt(x/2.2))  1
AGauss        1  266.25  -29.05   7  3 82 4.3e-6*TMath::Gaus(x,67.9,2.1)         1

MonoPoint     1  266.20  -29.15                               1  40   0.000010   1 
MonoDisk      2  266.30  -29.15 0.05                          1  50   0.000015   1     
MonoFile      5  $(NUSIM)/resource/examples/Tycho/Tycho.fits  1  60   0.000020   1

AllInOne      6  $(NUSIM)/resource/examples/Tycho/Tycho.3Ddat                    1

CrabPulsar    1  22.0 83.6  4  3 82 2 0.36  2 $(NUSIM)/res/examp/Crab/Crab.lgt true
\end{verbatim}
}

Each line represents a source (due to tex file formating one line might look as multiple in your printout) and each line consists of up to four sections:
\begin{enumerate}
\item The source name
\item The beam options
\item The spectral and flux options
\item The light curve options
\end{enumerate}

\subsubsection{Source name}

The first column is always the name of the source. It must not contain any spaces.

\subsubsection{Beam}

In general, different beam and spectral options result in different number of columns.
Therefore the beam and spectral options section is always preceeded by a number representing the beam type:

The beam options are:
\begin{enumerate}
\item FarFieldPoint
\item FarFieldDisk
\item NearFieldPoint --- do not use
\item NearFieldBeam --- do not use
\item FarFieldFitsFile
\item FarFieldNormalizedEnergyPositionFluxFunction
\end{enumerate}
Attention: Only the ones relevant for astrophysics are accessible via this input options file.

The far field point source is followed by two numbers: (1) RA in deg, (2) DEC in deg.

The far field disk source is followed by three numbers: (1) RA in deg, (2) DEC in deg, (3) The extend (radius) in deg.

The far field fits file is followed by a file name. Currently it must not contain any spaces! You can use \$(NUSIM) to represent the NuSIM directory: (1) File name.

The far field normalized combined energy-spectrum-flux function is only followed by file name since the file contains already all other options: (1) File name. Currently it must not contain any spaces! You can use \$(NUSIM) to represent the NuSIM directory


\subsubsection{Spectrum and flux}

The spectral options are:
\begin{enumerate}
\item Monoenergetic
\item Linear
\item PowerLaw
\item BrokenPowerLaw
\item FileDifferentialFlux
\item BlackBody
\item NormalizedFunctionInPhPerCm2PerSPerKeV
\item NormalizedEnergyPositionFluxFunction --- not required since redundant with FarFieldNormalizedEnergyPositionFluxFunction
\end{enumerate}


The monoenergetic option is followed by 2 numbers: 
(1) Energy in keV, 
(2) Flux in ph/cm2/s.

The linear option is followed by 3 numbers: 
(1) Minimum energy in keV, 
(2) Maximum energy in keV, 
(3) Flux in ph/cm2/s within the given spectral band.

The powerlaw option is followed by 4 numbers: 
(1) Minimum energy in keV, 
(2) Maximum energy in keV, 
(3) Photon index, 
(4) Flux in ph/cm2/s within the given spectral band.

The broken powerlaw option is followed by 6 numbers: 
(1) Minimum energy in keV,  
(2) Maximum energy in keV, 
(3) Break energy in keV, 
(4) Photon index low, 
(5) Photon index high, 
(6) Flux in ph/cm2/s within the given spectral band.

The file with the differential flux is given by only one file name and one flux: 
(1) File name. Again it must not contain any spaces! You can use \$(NUSIM) to represent the NuSIM directory.  The format is described in \ref{label:spectrumfile}.
(2) Flux in ph/cm2/s within the given spectral band in the file.

The black body option is followed by 4 numbers: 
(1) Minimum energy in keV, 
(2) Maximum energy in keV, 
(3) Temperature energy in keV, 
(4) Flux in ph/cm2/s within the given spectral band.

The normalized function in ph/cm2/s/keV is given by two number and a string (no flux required!): 
(1) Minimum energy in keV, 
(2) Maximum energy in keV, 
(3) Function string.
The string represents a function and must not contain any spaces!
You can use all C/C++ (pow, exp, sqrt, cos, sin, etc.) and all ROOT functions, such as TMath:Gaus(...), etc.


\subsubsection{Light curve}

The light curve options are:
\begin{enumerate}
\item Flat light curve
\item Light curve from file
\end{enumerate}

The flat light curve has no additional entries.

The option light curve from file has two entries:
(1) File name. Again it must not contain any spaces! You can use \$(NUSIM) to represent the NuSIM directory. The format is described in \ref{label:lightcurvefile}.
(2) Boolean ("true" or "false") indicating if the light curve repeats (e.g. for pulsars)

\section{For the pointing engine \label{inputformats:pointing}}

A possible input for the pointing is a pointing pattern file with the suffix pat.
It looks like this (from the Tycho example directory):
\begin{verbatim}
RD 6.29 64.115 0 10
RD 6.35 64.115 6 10
RD 6.41 64.115 12 10
RD 6.29 64.1375 18 10
RD 6.35 64.1375 24 10
RD 6.41 64.1375 30 10
RD 6.29 64.16 36 10
RD 6.35 64.16 42 10
RD 6.41 64.16 48 10
\end{verbatim}

The first number represents the right ascension in degree, the second number the declination in degree, the third number the Roll of the space craft in degree and the final number the relative or absolute observation time.


\section{For the background engine\label{label:hotpixelformat}}

The only user modifiable file in the background engine is the hot-pixel file.
One line in the file starting with "HP" is one hot pixel entry. The columns of the line have the following meaning:

\subsubsection{Format}

\begin{enumerate}
\item Telescope ID: 1, 2, or "-" if random
\item Detector ID: 1, 2, 3, 4, or "-" if random
\item x-position: in mm within wafer, or "-" if random
\item y-position: in mm within wafer, or "-" if random
\item z-position: in mm within wafer, or "-" if random
\item Start time in sec in NuSIM time (not absolute); if "-" then it is randomly determined by on and off duration 
\item On duration (after first event, so that always one event is created even if the duration is zero)
\item On duration uncertainty (Gaussian, 1 sigma) in sec
\item Off duration in sec, "-" means infinity, i.e. pixel is only on once
\item Off duration uncertainty (Gaussian, 1 sigma) of hotpixel in sec
\item Choose random pixel after off ("true"/"false"): if true, switch telescope, detector and position for the next firing sequence
\item Firing frequency in Hz (individual start times are Poisson distributed)
\item Firing frequency uncertainty (Gaussian, 1 sigma) in Hz
\item Mean energy in keV
\item Energy uncertainty (Gaussian, 1 sigma) in keV
\end{enumerate}

\subsubsection{Examples}

\noindent Random noise: Choose a random pixel, which only fires once
\begin{verbatim}
HP - - - - - - 0 0 0.01 0.02 true 1.0 0.0 2.5 2
\end{verbatim}

\noindent One pixel misbehaves from time-to-time for a certain amount of time
\begin{verbatim}
HP 2 3 6 -4 0.5 - 25 40 200 200.0 false 100 200.0 2.5 2
\end{verbatim}

\noindent A random pixel behaves badly for a certain amount of time and a different one after the next firing sequence
\begin{verbatim}
HP - - - - - - 100 200 100 200 true 100 200.0 2.5 2
\end{verbatim}

